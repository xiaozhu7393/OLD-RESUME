%% start of file `template-zh.tex'.
%% Copyright 2006-2013 Xavier Danaux (xdanaux@gmail.com).
%
% This work may be distributed and/or modified under the
% conditions of the LaTeX Project Public License version 1.3c,
% available at http://www.latex-project.org/lppl/.


\documentclass[11pt,a4paper,sans]{moderncv}  
\usepackage[utf8]{inputenc}                  
\usepackage{CJKutf8}
\usepackage{info}                 

\setlength{\hintscolumnwidth}{3cm}           
\name{黄}{峰达}
\title{Phodal Huang}                      
\phone[mobile]{+086 18209219631}              
\email{h@phodal.com}                    
\homepage{https://www.phodal.com}                   
\photo[64pt][0.2pt]{avatar}                   

\begin{document}
\begin{CJK}{UTF8}{gbsn}                        
\maketitle

\section{{工作经历}}
\cventry{2014.7-今}{{在职}}{软件开发工程师}{ThoughtWorks(西安、深圳)}{现深圳}{}{}

\section{{项目经历}}
\cventry{2013.12-2016.04}{ThoughtWorks - 澳洲某大型房地产搜索网站}{}{开发人员}{技术负责人}{
\begin{tightitemize}
        \item 项目描述: 开发网站的桌面版、移动版、API、服务组件,并用微服务架构与响应式设计对系统重构。
        \item 语言: Java、JavaScript、Ruby、Scala
        \item 技术栈:  Spring MVC、Node.js、ElastichSearch、Backbone、React、微服务、AWS
        \item 主要贡献: 作为开发人员,负责网站主站维护、移动网站开发。作为技术负责人,负责对网站进行重构,并采用新架构进行设计。并对新人进行指导和培训,协助团队进行更好的进行敏捷开发及软件工程实践。
\end{tightitemize}}
\cventry{2016.04-今}{ThoughtWorks - 某大型国有银行}{}{移动端负责人}{前端负责人}{
	\begin{tightitemize}
		\item 项目描述: 某大型国有银行同业金融合作平台,开发网站前端及基于 Ionic 与 React Native 的 APP。
		\item 语言: JavaScript、Java、Objective-C
		\item 技术栈:  Angular.js、Angular 4、Ionic、Cordova、React Native
		\item 主要贡献: 1. 开发基于某聊天 SDK 的 Cordova 插件。2. 负责 APP 从开发到 iOS、Android上线改版等工作。3. 负责结合 Cordova WebView 与 React Native 的设计、实施、插件编写。
	\end{tightitemize}}

\section{技能}

\subsection{语言}
\cvcomputer{精通}{JavaScript}
           {熟练}{Python、Ruby、TypeScript、Java}

\subsection{框架}           
\cvcomputer{精通}{Django、Angular.js、Ionic}           
           {熟练}{Angular 2、React、React Native、Spring}
		
\section{社区}
\cvitemwithcomment{个人博客}{ \href{https://www.phodal.com/}{https://www.phodal.com}}{日均PV:500}
\cvitemwithcomment{GitHub}{ \href{https://github.com/phodal}{https://github.com/phodal}}{全球Top 100}
\cvitemwithcomment{InfoQ}{ http://www.infoq.com/cn/author/黄峰达 }{社区编辑}
\cvitemwithcomment{CSDN}{ \href{http://blog.csdn.net/phodal}{http://blog.csdn.net/phodal}}{前端博客专家}

\section{开源项目}
\cventry{2014}{ 移动开发学习框架 Lettuce \href{https://github.com/phodal/lettuce}{https://github.com/phodal/lettuce}}{}{}{}{}
\cventry{2015}{ 展示、幻灯片框架 EchoesWorks \href{https://github.com/phodal/echoesworks}{https://github.com/phodal/echoesworks}}{}{}{}{}
\cventry{2015}{ 基于 Virtual DOM 测试辅助框架 Luffa \href{https://github.com/phodal/luffa}{https://github.com/phodal/luffa}}{}{}{}{}
\cventry{2015-2016}{ 物联网服务端框架 Lan \href{https://github.com/phodal/lan}{https://github.com/phodal/lan}}{}{}{}{}
\cventry{2015-今}{跨平台Web开发学习应用—Growth,1.0-2.0(Ionic,\href{https://github.com/phodal/growth-v2}{https://github.com/phodal/growth-v2})及 3.0(React Native,\href{https://github.com/phodal/growth}{https://github.com/phodal/growth}) 的开发(可从 App Store 或 Google Play、应用宝等 Android 应用商店下载)}{}{}{}{日活 300}

\section{其他} 
\cventry{电子工业出版社}{《自己动手设计物联网系统》}{作者}{}{}{}
\cventry{电子工业出版社}{《全栈应用开发:精益实践》}{作者}{}{}{}
\cventry{机械工业出版社}{《物联网实战指南》}{中文版译者之一}{}{}{}
\cventry{机械工业出版社}{《Arduino编程:实现梦想的工具和技术》}{中文版译者之一}{}{}{}
\cventry{PACKT 出版社}{《Learning Internet of Things》、《Smart IoT Projects》、《Expert Angular》《Angular Service》、《Getting Started with Angular 第二版》等}{英文版技术审阅}{}{}{}

\section{开源电子书}           

\cventry{2014}{《教你设计物联网系统》GitHub \href{https://github.com/phodal/designiot}{https://github.com/phodal/designiot} }{}{}{}{}
\cventry{2015}{《GitHub 漫游指南》GitHub \href{https://github.com/phodal/github}{https://github.com/phodal/github} }{}{}{}{}
\cventry{2015-2016}{《全栈增长工程师指南》GitHub \href{https://github.com/phodal/growth-ebook}{https://github.com/phodal/growth-ebook} }{}{}{}{}
\cventry{2016}{《全栈增长工程师实战》GitHub \href{https://github.com/phodal/growth-in-action}{https://github.com/phodal/growth-in-action} }{}{}{}{}
\cventry{2017}{《我的职业是前端工程师》GitHub \href{https://github.com/phodal/fe}{https://github.com/phodal/fe} }{}{}{}{}
\cventry{2017}{《Serverless 应用开发指南》GitHub \href{https://github.com/phodal/serverless}{https://github.com/phodal/serverless} }{}{}{}{}

\section{{教育背景}}
\cventry{2010.9-2014.6}{{本科}}{电子信息工程}{西安文理学院}{}{}

\section{{实习经历}}
\cventry{2013.12-2014.6}{{实习}}{软件开发工程师}{ThoughtWorks(西安)}{}{}{}

\renewcommand{\listitemsymbol}{-}             % 改变列表符号

% 来自BibTeX文件但不使用multibib包的出版物
%\renewcommand*{\bibliographyitemlabel}{\@biblabel{\arabic{enumiv}}}% BibTeX的数字标签
\nocite{*}
\bibliographystyle{plain}
\bibliography{publications}                    % 'publications' 是BibTeX文件的文件名

% 来自BibTeX文件并使用multibib包的出版物
%\section{出版物}
%\nocitebook{book1,book2}
%\bibliographystylebook{plain}
%\bibliographybook{publications}               % 'publications' 是BibTeX文件的文件名
%\nocitemisc{misc1,misc2,misc3}
%\bibliographystylemisc{plain}
%\bibliographymisc{publications}               % 'publications' 是BibTeX文件的文件名

\clearpage\end{CJK}
\end{document}


%% 文件结尾 `template-zh.tex'.
