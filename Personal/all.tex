%% start of file `template-zh.tex'.
%% Copyright 2006-2013 Xavier Danaux (xdanaux@gmail.com).
%
% This work may be distributed and/or modified under the
% conditions of the LaTeX Project Public License version 1.3c,
% available at http://www.latex-project.org/lppl/.


\documentclass[11pt,a4paper,sans]{moderncv}  
\usepackage[utf8]{inputenc}                  
\usepackage{CJKutf8}
\usepackage{info}                 

\setlength{\hintscolumnwidth}{3cm}           
\name{黄}{峰达}
\title{http://about.phodal.com}                      
\address{陕西省西安市太白南路168号}{710065}            
\phone[mobile]{+18209219631}              
\email{h@phodal.com}                    
\homepage{www.phodal.com}                   
\extrainfo{Power by \LaTeX  }                   
\photo[64pt][0.2pt]{picture}                   

\begin{document}
\begin{CJK}{UTF8}{gbsn}                        
\maketitle

\section{{教育经历}}
\tlcventry{2010.9}{2014.6}{{本科在读}}{电子信息工程}{西安文理学院}{西安}{}


\subsection{{在校情况}}
\tllabelcventry{2010.9}{2011.1}{2010.9-2011.1}{墨颀多媒体文学社•社长}{西安文理学院}{}{}{
\begin{tightitemize}
	\item 社团创始人之一兼第一任社长
\end{tightitemize}}

\tllabelcventry{2012.8}{2012.12}{2012.12}{挑战杯科技制作大赛特等奖}{西安文理学院}{}{}{
\begin{tightitemize}
	\item 名称:《基于Android与Arduino的远程智能家居控制系统》 
	\item 职责: Android客户端编写、服务端编程、网站制作及运营、项目规划
\end{tightitemize}}
\tllabelcventry{2012.10}{2013.6}{2013.6}{第九届 陕西省 大学生课外学术科技作品竞赛二等奖}{}{}{}{
\begin{tightitemize}
	\item 名称:《远程控制的智能云家居系统》 
	\item 职责: Android客户端编写、服务端编程、网站制作及运营、项目规划、文档编写
\end{tightitemize}}
\tllabelcventry{2013.1}{2013.7}{2013.7}{第八届全国大学生“飞思卡尔”杯}{西部赛区 电磁组三等奖}{}{}{}

\section{工作经验}
\subsection{网站建设}
\tllabelcventry{2012.6}{2012.10}{2012.10}{福建荣宇有限公司}{官网升级}{网站前端}{}{
\begin{tightitemize}
        \item 网站前端设计 后台运维 
        \item 公司主页: {\httplink{www.fjrongyu.com/}}
\end{tightitemize}}
\tllabelcventry{2013.7}{2013.9}{2013.7}{陕西省福建商会}{官网迁移}{网站前端}{}{
\begin{tightitemize}
        \item 网站前端设计 后台运维
        \item 公司主页: {\httplink{www.sxsfjsh.com/}}
\end{tightitemize}}

\section{{项目经历}}
\tllabelcventry{2012.10}{2013.6}{2012-2013}{远程控制的智能云家居系统}{项目负责人}{}{}{
\begin{tightitemize}
        \item 关键技术:Django, RESTful API,Android ADK,物联网, Ajax,XBEE
        \item 编程语言:Python,Java,C/C++,Processing,Javascript
        \item 采用RESTful Web服务来实现网络控制与传输。
        \item 采用Ajax技术实现数据实时更新。
        \item 采用XBee来实现无线传感器网络。
        \item wiki主页:{\httplink{api.phodal.com/wiki/}}
\end{tightitemize}}
\tllabelcventry{2012}{2013.4}{2012-2013}{{远程控制的智能云家居系统Android客户端}}{主要开发人员}{编程语言: Java}{}{
\begin{tightitemize}
        \item 关键技术: RESTful,Java,
        \item 实现手机对用电器控制及与单片机通讯
        \item 项目主页: {\httplink{code.google.com/p/home-anywhere/}}
\end{tightitemize}}
\tllabelcventry{2012}{0}{2012-2013}{{远程控制的智能云家居系统服务端与网站}}{主要开发人员}{}{}{
\begin{tightitemize}
        \item 关键技术:HTTP, 协议分析, RESTful,Ajax,jQuery Mobile
		\item 编程语言:Bash,Javascript,Python,
        \item 采用Ajax技术实现数据实时更新
        \item 采用jQuery及jQuery Mobile来实现用户界面优化及多平台访问
        \item 项目主页: {\httplink{api.phodal.com}}
\end{tightitemize}}

\subsection{研究、开发经历}
\tlcventry{02008.11}{0}{GNU/Linux相关项目、研究}{}{}{}{
\begin{tightitemize}
\item Linux服务器(CentOS)运行及维护 
\item 使用LFS编译自己的Linux系统
\item Linux+Apache+MySQL+PHP Linux+Nginx+MySQL+PHP+Python LAMP LNMP架设
\item Linux内核研究及其编译
\item GNU/Linux版本Ubuntu Linux,Centos Linux,OpenSUSE Linux,OpenWRT Linux,Mint Linux,Debian Linux
\item Linux C、Linux Shell开发经验
\end{tightitemize}}

\tlcventry{2010.11}{0}{网络相关项目、研究}{}{}{}{
\begin{tightitemize}
 \item 使用jQuery+Timerliner+Bootstrap完成的个人网页简历 {\httplink{about.phodal.com}}
 \item 使用PHP与Laravel做的CMS {\httplink{www.xianuniversity.com}}
 \item 使用Node.j和WebSocket实现了简易在线聊天系统  {\httplink{socket.phodal.com}}
 \item 使用Python、Django、 Bootstrap维护的个人网站 {\httplink{www.phodal.com}}
 \item 个人博客、SEO优化、微数据(Microdata) {\httplink{blog.phodal.com}}
 \item 弘知书社客户端及网站建设 {\httplink{www.hongzhishushe.com}}
\end{tightitemize}}

\tlcventry{2011.5}{0}{嵌入式相关项目、研究}{}{}{}{
\begin{tightitemize}
 \item 使用QT4完成移动蜂窝模型通信系统构建
 \item 使用flex和bison ,antlr完成了简单的翻译器
 \item 使用Freescale K60实现寻线小车,使用Freescale XS128实现飞思上尔电磁车
 \item 移植GCC for ARM到Android G1 {\httplink{blog.csdn.net/phodal}}
 \item 使用OpenWRT Linux实现路由器上运行Python,运行RESTful服务
 \item 计算机语言ADA、C++在AVR芯片上的使用
 \item uCOS II在STC89C52芯片上的多任务开发
\end{tightitemize}}

\tlcventry{2010.11}{0}{Github开源项目}{}{}{}{
\begin{tightitemize}
 \item Python Alioss包维护 {\httplink{github.com/gmszone/alioss}} 
 \item 代码而成的简历 {\httplink{github.com/gmszone/RESUME}}
 \item 数据可视化  {\httplink{github.com/gmszone/DataVisual}}
\end{tightitemize}}
\tlcventry{2010.11}{0}{Github、Google CODE托管项目}{}{}{}{
\begin{tightitemize}
 \item 网站制作集合 {{code.google.com/p/phodal-web-development/}} 
 \item PHP学习 {\httplink{github.com/gmszone/learingphp}}
 \item emacs配置及Django学习 {\httplink{code.google.com/p/phodal-gae-django-first}}
 \item 其他参见 {\httplink{github.com/gmszone}}
\end{tightitemize}}

\section{技能}

\subsection{开发}
\cvcomputer{语言}{C/C++,Python,JavaScript, Lisp, C\#,Java,PHP,Ruby}
           {数据库}{MySQL, MongoDB, SQLite}
           
\cvcomputer{框架}{Django, Pomelo, Bootstrap, jQuery,jQuery Mobile, Cocos2d-X,Laravel,}
     		{Web}{DIV+CSS3, Nodejs, jQuery, AJAX, RESTful, Socket,Angularjs,D3,}     

\subsection{工具}
\cvcomputer{办公}{LibreOffice, Microsoft Office, \LaTeX, Graphviz}
           {操作系统}{Mac OS X, GNU/Linux(penSUSE, Centos), Windows}
\cvcomputer{Design}{InkScape, Photoshop, GIMP, Dot, Texworks}
           {编辑器}{Emacs, VIM, Sublime Text,  Notepad++, Eclipse, Aptana,QT}

\subsection{嵌入式}           
\cvcomputer{芯片}{Atmega family, 51, STM32, MSP430, Freescale K60, Freescale XS128, ARM}
           {开发环境}{IAR for ARM, Keil C51, ADS, Arduino IDE, Android Sudio,Eclipse,Energia}
\cvcomputer{EOS}{$\mu$C/OS $\Pi$,openWRT Linux,Linux for ARM}
		{其他}{OpenCV, TTS, Raspberry Pi, MATLAB}
		
\section{博客}
\cvitemwithcomment{CSDN}{http://blog.csdn.net/phodal}{PHP}
\cvitemwithcomment{Phodal}{http://www.phodal.com/blog}{Django}
\cvitemwithcomment{个人}{http://blog.phodal.com}{WORDPRESS}

\section{语言技能}
\cvitemwithcomment{英语}{}{精通读写}
\cvitemwithcomment{闽南语}{}{精通听说}

\section{个人信息}
\cvitem{特长}{\small 计算机相关,互联网相关,网站创作、设计及运维,嵌入式系统设计,架构设计, DIY,设计,开源}
\cvitem{职业技能}{\small 精通GNU/Linux,能在不同的平台编程,精通Emacs,熟练使用Vim,同时能使用Eclipse/及Visual studio编程,熟练使用Android与单片机通讯,不同单片机设备间的通讯。精通不同平台的排版及文档创作,精通使用office,以及tex/latex/miktex/生成文档,dot生成文档,系统构架分析及设计。}
\cvitem{自我评价}{\small 涉猎甚广,除计算机外,所学不精。永无休止的求知欲和好奇心,擅长辞令,诙谐幽默。更喜欢各种研究性的、独立的、创造性的、与观念或事务打交道的工作;比较喜欢现实性的、操作性的、户外的、与事务打交道的工作;还喜欢艺术性、创造性、独立性、与观念打交道的工作。}
\cvitem{兴趣爱好}{\small 计算机,漫游互联网,接触各种电子产品,喜欢玩不同手机,策略类游戏,DIY,电子diy,羽毛球,网球,爬山,骑自行车。素描,喜欢科幻片,旅游——到不同的地方。侦探小说,三国演义,福尔摩斯。喜欢中国风歌曲,写诗。}

\renewcommand{\listitemsymbol}{-}             % 改变列表符号

% 来自BibTeX文件但不使用multibib包的出版物
%\renewcommand*{\bibliographyitemlabel}{\@biblabel{\arabic{enumiv}}}% BibTeX的数字标签
\nocite{*}
\bibliographystyle{plain}
\bibliography{publications}                    % 'publications' 是BibTeX文件的文件名

% 来自BibTeX文件并使用multibib包的出版物
%\section{出版物}
%\nocitebook{book1,book2}
%\bibliographystylebook{plain}
%\bibliographybook{publications}               % 'publications' 是BibTeX文件的文件名
%\nocitemisc{misc1,misc2,misc3}
%\bibliographystylemisc{plain}
%\bibliographymisc{publications}               % 'publications' 是BibTeX文件的文件名

\clearpage\end{CJK}
\end{document}


%% 文件结尾 `template-zh.tex'.
